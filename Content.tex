% section color
\definecolor{darkblue}{RGB}{0, 0, 139}

% Define a new column type that applies the background color
\newcolumntype{C}{>{\columncolor{shadecolor}}X}

\begin{table}[h]
\centering
\renewcommand{\arraystretch}{1.7}
\arrayrulecolor{black} % Set the color of the table lines to black
\begin{tabularx}{\textwidth}{|C|C|}
\hline
\textcolor{darkblue}{\textbf{Project Title}} & Reincarnation of construction products and materials by slowing down and extending life cycles. \\ \hline
\textcolor{darkblue}{\textbf{Project Acronym}} & Reincarnate \\ \hline
\textcolor{darkblue}{\textbf{Grant Agreement No}} & 101056773 \\ \hline
\textcolor{darkblue}{\textbf{Instrument}} & Research \& Innovation Action \\ \hline
\textcolor{darkblue}{\textbf{Topic}} & HORIZON-CL4-2021-TWIN-TRANSITION-01-11 \\ \hline
\textcolor{darkblue}{\textbf{Start Date of Project}} & June 1, 2022 \\ \hline
\textcolor{darkblue}{\textbf{Duration of Project}} & 48 months \\ \hline
\end{tabularx}
%\caption{Project Details}
\end{table}
%--------------------------------------------------------------------------------------
\begin{table}[ht!]
\centering
\renewcommand{\arraystretch}{1.7} % Adjust the value as needed
\arrayrulecolor{black} % Set the color of the table lines to black
\begin{tabularx}{\textwidth}{|C|C|}
\hline
\textcolor{darkblue}{\textbf{Name of the Deliverable}} & An ontology for circular management of buildings \\ \hline
\textcolor{darkblue}{\textbf{Number of the Deliverable}} & D1.1 \\ \hline
\textcolor{darkblue}{\textbf{Related WP Number and Name}} & WP1 \\ \hline
\textcolor{darkblue}{\textbf{ Related Task Number and Name}} & Task 1: Ontologies information models and inspection \\ \hline
\textcolor{darkblue}{\textbf{Deliverable Dissemination Level}} & R — Document report PU- Public \\ \hline
\textcolor{darkblue}{\textbf{Deliverable Due Date}} & 30.11.2023 \\ \hline
\textcolor{darkblue}{\textbf{Deliverable Submission Date}} & xxxx \\ \hline
\textcolor{darkblue}{\textbf{Task Leader/Main Author}} & Samaneh Rezvani (DMO), Fatemeh Asgharzadeh (DMO), Maurijn Neumann (DMO) \\ \hline
\textcolor{darkblue}{\textbf{Contributing Partners}} & Karin Wannerberg (RagnSell), Christina Stålhandske (RagnSell), Iwona Wagner (CEMX), Lukasz Szabat (CEMX) \\ \hline
\textcolor{darkblue}{\textbf{Reviewer(s)}} & Timo Hartmann (TUB), Sabine (BAM) \\ \hline
\end{tabularx}
%\caption{Deliverable Specifics}
\end{table}
\newpage
%--------------------------------------------------------------------------------------
\section*{Abstract}
Currently, there is no widely agreed-upon set of terms and ideas related to the circular economy. \\
%--------------------------------------------------------------------------------------
\textcolor{darkblue}{\textbf{Keywords:\\}}
Circular Economy, Machine Learning,..
%--------------------------------------------------------------------------------------
\begin{center}
    \textcolor{darkblue}{\textbf{Revisions}}
\begin{table}[h]
\centering
\renewcommand{\arraystretch}{1.3}
\arrayrulecolor{black} % Set the color of the table lines to black
\begin{tabularx}{\textwidth}{|C|C|C|C|}
\hline
\textcolor{darkblue}{\textbf{Version}} & \textcolor{darkblue}{\textbf{Submission date}} & \textcolor{darkblue}{\textbf{Comments}} & \textcolor{darkblue}{\textbf{Author}}\\ \hline
v0.1 & 7.11.2023 & & \\ \hline
v0.2 &  & &  \\ \hline
... &  & &  \\ \hline
... & &  &  \\ \hline
... & & &  \\ \hline
\end{tabularx}
%\caption{Project Details}
\end{table}
\end{center}
%--------------------------------------------------------------------------------------
\begin{center}
    \textcolor{darkblue}{\textbf{Disclaimer}}
\end{center}

\begin{tcolorbox}[width=0.99\textwidth, colframe=black, colback=white]
\textcolor{darkblue}{
This document is provided with no warranties whatsoever, including any warranty of merchantability, non-infringement, fitness for any particular purpose, or any other warranty with respect to any information, result, proposal, specification or sample contained or referred to herein. Any liability, including liability for infringement of any proprietary rights, regarding the use of this document or any information contained herein is disclaimed. No license, express or implied, by estoppel or otherwise, to any intellectual property rights is granted by or in connection with this document. This document is subject to change without notice. Reincarnate has been financed with support from the European Commission. This document reflects only the view of the author(s) and the European Commission cannot be held responsible for any use which may be made of the information contained.
}
\end{tcolorbox}
%--------------------------------------------------------------------------------------
\section*{Acronyms and definitions}
\begin{table}[h]
\centering
\renewcommand{\arraystretch}{1.3}
\arrayrulecolor{black} % Set the color of the table lines to black
\begin{tabularx}{\textwidth}{|C|C|}
\hline
 \textcolor{darkblue}{\textbf{Acronym}} & \textcolor{darkblue}{\textbf{Meaning}}\\ \hline
XD &	eXtreme Design \\ \hline
MOMo	&Modular Ontology Modelling \\ \hline
... & ...\\ \hline
\end{tabularx}
%\caption{Project Details}
\end{table}
\newpage
%--------------------------------------------------------------------------------------
\section*{Reincarnate project}
The average lifespan of a building is 39 years — in Europe, it is only 25-30 years — and the main reason for demolition is obsolescence. This is why there is a large amount of construction and demolition waste (CDW) — representing approximately 25-30\% of all waste in Europe —, in addition to that generated in current construction works.

The recycling rate for CDW is relatively high (above 75\%). This activity generated \$126.89 billion in 2019 — Europe contributed the largest share, almost two-fifths of the total global market — and is projected to reach \$149.19 billion by 2027. Unfortunately, many of the most valuable materials in CDW cannot be meaningfully separated and end up in landfills.
This helps to get an idea of the efficiency potential for climate neutrality that exists in construction. 

Reincarnate aims at advancing circular economy practices within the European construction industry and enabling to significantly maximize the life cycle of buildings, construction products, and materials, reduce CDW by 80\%, increase the reusability of buildings, construction products and materials and, as a result, lower the sector's emissions by 70\%.

As a result of these actions, Reincarnate will significantly advance circular economy practices within the European construction industry.
First, it will create a Circular Potential Information Management (CP-IM) platform and a set of innovations to use it.  These solutions will draw upon emerging digital technologies, such as digital twin representation, artificial intelligence, and robotic automation.

Three empirically proven social science insights will allow fostering widespread adoption of reused high-quality construction products and materials, and business eco-system development frameworks to combine actors within sustainable value chains. All innovations will be demonstrated on eleven selected real-world projects and value chains. Furthermore, business process guidelines and an e-learning platform will be developed to drive the dissemination and exploitation of the Reincarnate results.
%--------------------------------------------------------------------------------------
\newpage
\tableofcontents % This command generates the table of contents

\newpage
\listoffigures % List of Figures

\newpage
\listoftables % List of Tables

\newpage
%--------------------------------------------------------------------------------------
\section*{Executive Summary (4p, Christoph, Sabine)}
\todo[inline]{A concise overview of the report's goals, methodology, key findings, and conclusions.}


%--------------------------------------------------------------------------------------
\section{Definitions and Interpretations}

\textbf{\textcolor{darkblue}{Construction Material}}: Refers to the raw substances or elements used in creating building structures, such as concrete, steel, wood, and composites. These materials are fundamental to the construction process, and their properties directly impact the overall quality, durability, and performance of the built environment.

\textbf{\textcolor{darkblue}{Construction Component}}: Involves manufactured or fabricated elements that are part of a larger structure, like beams, columns, panels, or window units. These components are integral to the structural and functional aspects of buildings and infrastructure.


\textbf{\textcolor{darkblue}{Building System}}:
Refers to the comprehensive assembly of interconnected components that function together to provide essential services or perform specific functions within a building. This includes systems like HVAC (Heating, Ventilation, and Air Conditioning), electrical systems, plumbing, and structural frameworks. Building systems are designed to work in harmony to ensure the building's functionality, comfort, safety, and energy efficiency.\\


\textbf{\textcolor{darkblue}{Data-Driven Method}}:
A data-driven method refers to an approach that relies on data analysis to inform decision-making, predictions, or understanding of phenomena. In the context of construction materials and products, these methods often utilize statistical, machine learning, or AI algorithms to analyze data and draw insights about material conditions and durability.\\

\textbf{\textcolor{darkblue}{Condition Assessment}}:
Condition assessment in construction refers to the process of evaluating the state or health of a material, product, or system. It often involves the analysis of various indicators, such as physical properties, performance data, and degradation over time, to determine current conditions and predict future performance.\\

\textbf{\textcolor{darkblue}{Durability}}:
Durability in the context of building materials refers to the ability of a material or product to withstand environmental conditions and operational stresses over time without significant deterioration. It's a key factor in assessing the long-term performance and sustainability of construction materials.\\

\textbf{\textcolor{darkblue}{Non-destructive testing}}:
Non-Destructive Testing in construction is a range of techniques used to evaluate the properties, composition, and integrity of materials and structures without causing damage.\\

%--------------------------------------------------------------------------------------
\section{Introduction and Background (10-15p, Zia)}
\todo[inline, color=green!40]{Combines the context, objectives, and a wide literature review relevant to construction material models, nondestructive testing, and digital workflows in building inspection.}
\todo[inline, color=green!40]{	The first draft of the Excel Table created, crossing materials, NDT methods, and data-driven methods, has now to be brought in the perspective of this dev}
%--------------------------------------------------------------------------------------
\subsection{Concrete}

\begin{table}[h!]
\centering
\begin{tabularx}{\textwidth}{|X|X|X|}
\hline
\textbf{Mechanism} & \textbf{Consequence on concrete} & \textbf{What is looked for?} \\
\hline
Overloading & Damage cracking & 
\begin{itemize}[leftmargin=*, nosep, after=\strut]
    \item if distributed damage: crack density residual stiffness and strength
\end{itemize} \\
\hline
Restraining effects (temperature shrinkage) & & 
\begin{itemize}[leftmargin=*, nosep, after=\strut]
    \item if localized cracking: location width depth
\end{itemize} \\
\hline
Freeze–thaw cycles & Scaling spalling delamination & 
\begin{itemize}[leftmargin=*, nosep, after=\strut]
    \item delaminating areas
    \item depth of delamination
\end{itemize} \\
\hline
Fire & Strength decrease spalling & 
\begin{itemize}[leftmargin=*, nosep, after=\strut]
    \item depth reached by fire effects
    \item residual strength at various depths
\end{itemize} \\
\hline
Abrasion–erosion & Material loss & 
\begin{itemize}[leftmargin=*, nosep, after=\strut]
    \item residual strength of surface layer
\end{itemize} \\
\hline
Carbonation & Increase in density depassivation of steel thus rebar corrosion & 
\begin{itemize}[leftmargin=*, nosep, after=\strut]
    \item carbonation depth
    \item if corrosion: localization of active corrosion areas corrosion rate
\end{itemize} \\
\hline
Chloride attack & Rebar corrosion & 
\begin{itemize}[leftmargin=*, nosep, after=\strut]
    \item chloride content chloride profile
    \item if corrosion: localization of active corrosion areas corrosion rate
\end{itemize} \\
\hline
Alkali–aggregate reaction & Internal expansion generalized cracking & 
\begin{itemize}[leftmargin=*, nosep, after=\strut]
    \item potential for future volume change
\end{itemize} \\
\hline
Sulfate attack & & 
\begin{itemize}[leftmargin=*, nosep, after=\strut]
    \item residual stiffness and strength
\end{itemize} \\
\hline
Leaching & Cement paste dissolution increase in porosity & 
\begin{itemize}[leftmargin=*, nosep, after=\strut]
    \item residual strength porosity
\end{itemize} \\
\hline
Ammonium nitrate attack & Deterioration of the cement paste spalling rebar corrosion & 
\begin{itemize}[leftmargin=*, nosep, after=\strut]
    \item depth of the attack
    \item if corrosion: localization of active corrosion areas corrosion rate
\end{itemize} \\
\hline
\end{tabularx}
\caption{Mechanisms of concrete deterioration and their evaluation.}
\label{tab:mechanisms}
\end{table}
%--------------------------------------------------------------------------------------

\subsection{Timber/Wood}



\begin{table}[ht!]
\centering
\renewcommand{\arraystretch}{1.3} % More space between rows
\small % Smaller font size for the table
\begin{tabularx}{\textwidth}{|X|X|X|}
\hline
\multicolumn{1}{|c|}{\textbf{Indirect factors}} & \multicolumn{2}{c|}{\textbf{Direct Factors}} \\ \hline
 & \textbf{Exogenous factors (environmental influences)} & \textbf{Endogenous factors (material-inherent resistance)} \\ \hline
Climate & Presence of species & Natural resistance \\
\hspace{5mm}Air & &\\
\hspace{5mm}Temperature & Decay types & \hspace{5mm}Wood species \\
\hspace{5mm}Precipitation & \hspace{5mm}Degradation mechanisms & \hspace{5mm}Extractives Position in the log \\
\hspace{5mm}Wind & Inoculum potential &  \\
\hspace{5mm}Relative humidity & Wood temperature & \hspace{5mm}Juvenile/adult wood \\
Construction/design & \hspace{5mm}Dynamics of conditions & \hspace{5mm}Reaction wood \\
Dimension & Wood moisture content & \hspace{5mm}Provenience \\
Shadow & \hspace{5mm}Nutrients & \hspace{5mm}Felling time \\
Distance to ground & Relationship between organisms & Storage \\
Orientation &  & \hspace{5mm}Density \\
Roofing &  & \hspace{5mm}Tylosis \\
 &  & Improved resistance \\
 &  & \hspace{5mm}Type of wood preservative \\
 &  & \hspace{5mm}Retention of wood preservative \\
 &  & \hspace{5mm}Penetration/distribution of wood preservative \\
 &  & \hspace{5mm}Wood modification \\
 &  & \hspace{5mm}Hydrophobation \\
 &  & \hspace{5mm}Coatings \\
 &  & \hspace{5mm}Biological wood protection \\
\hline
\end{tabularx}
\caption{Factors influencing wood durability and preservation.}
\label{table:factors}
\end{table}
%--------------------------------------------------------------------------------------
\subsection{Steel/Metal}
%--------------------------------------------------------------------------------------
\subsection{Masonry/Bricks}
%--------------------------------------------------------------------------------------


\section{Methodology and Data Integration (CP-IM workflow and ontology, max. 5p, Ben)}

\todo[inline, color=green!40]{Details the CP-IM information, nondestructive testing methods, data collection, and techniques for integrating nondestructive field data with destructive material testing data.}
\todo[inline, color=green!40]{- Role of ontologies in the data integration (use contents of NMR paper)}

%--------------------------------------------------------------------------------------

\section{Analysis and Development of Material Models (data-driven methods for ndt-data, a machine learning model that works on ndt-data input and calculates (future) condition status, 10p Zia)}

\todo[inline, color=green!40]{Focuses on analyzing chemical and mechanical properties, developing models for assessing material status and lifetime, and discussing challenges in data integration.}
\todo[inline, color=green!40]{- Adaptive Sampling Approach (Zia’s Paper)}
%--------------------------------------------------------------------------------------


\section{Digital Workflow and Field Applications (focus on a specific case study here, 3per case study in total 9p, 3p CEMEX, 3p RAS, HKP???, Zia, Christoph, Sabine)}	
\todo[inline, color=green!40]{Describes the integration of non-destructive testing into the digital workflow and its application in the Building Inspection and Valuation methodology, including case studies.}
\todo[inline, color=green!40]{- MFX, shows how such a case study is integrated in their software (who?)}
\todo[inline, color=green!40]{- Cemex has to report their case study and we contribute with possible NDT solutions (Zia, Sabine)}
\todo[inline, color=green!40]{ RAS (Zia, Christoph)}



%--------------------------------------------------------------------------------------


\section{Conclusions and Recommendations (4p, ALL)}	
Combines discussion of implications, suggestions for future research, and practical recommendations for industry practice.
%--------------------------------------------------------------------------------------
